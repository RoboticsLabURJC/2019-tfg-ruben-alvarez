\begin{thebibliography}{99}

    \bibitem{bib:navegadores}
    \textit{Navegadores web más empleados:}

    \url{https://www.mozilla.org/}
    
    \url{https://www.opera.com/}
    
    \url{https://www.google.com/intl/es/chrome/}
    
    \bibitem{bib:ice}
    \textit{Página oficial de ICE:}
    \url{https://zeroc.com/products/ice}
    
    \bibitem{bib:ros}
    \textit{Página oficial de ROS:}
    \url{https://www.ros.org/}
    
    \bibitem{bib:orca}
    \textit{Página oficial de ORCA:}
    \url{http://orca-robotics.sourceforge.net/}
    
    \bibitem{bib:orocos}
    \textit{Página oficila de OROCOS:}
    \url{https://orocos.org/}
    
    \bibitem{bib:secundaria}
    Comunidad de Madrid. %Quien lo escribe
    \textit{Asignatura de robótica en Secundaria.} %Nombre
    SIMO EDUCACIÓN, Octubre 2015. % Periodico y fecha de publicacion
    \url{https://n9.cl/7lqc}. % URL

    \bibitem{bib:primaria}
    \textit{Asignatura de robótica en Primaria:}
    \url{http://www.telemadrid.es/noticias/madrid/madrilenos-impartiran-Programacion-Robotica-Primaria-0-2158584125--20190914103816.html}

    \bibitem{bib:scratch}
    \textit{Página oficial de Scratch:}
    \url{https://scratch.mit.edu/}
    
    \bibitem{bib:lego}
    \textit{Página oficial de LEGO programming:}
    \url{https://www.lego.com/es-es/categories/coding-for-kids}
    
    \bibitem{bib:kodu}
    \textit{Página oficial de Kodu:}
    \url{https://www.kodugamelab.com}
    
    \bibitem{bib:snap}
    \textit{Página oficial de Snap!:}
    \url{https://snap.berkeley.edu}
    
    \bibitem{bib:appinventor}
    \textit{Página oficial de AppInventor:}
    \url{http://ai2.appinventor.mit.edu/}
    
    \bibitem{bib:makeblock}
    \textit{Página oficial de Makeblock:}
    \url{https://makeblock.es/}
    
    \bibitem{bib:arduino}
    \textit{Página oficial de Arduino:}
    \url{https://www.arduino.cc/}
    
    \bibitem{bib:docblockly}
    \textit{Documentación oficial de Blockly:}
    \url{https://developers.google.com/blockly/guides/create-custom-blocks/overview}
    
    \bibitem{bib:npm}
    \textit{Documentación oficial de NPM:}
    \url{https://www.npmjs.com/}
    
    \bibitem{bib:webpack}
    \textit{Documentación oficial de WebPack:}
    \url{https://webpack.js.org/}
    
    \bibitem{bib:grunt}
    \textit{Documentación oficial de Grunt:}
    \url{https://gruntjs.com/}
    
    \bibitem{bib:gulp}
    \textit{Documentación oficial de Gulp:}
    \url{https://gulpjs.com/}
    
    \bibitem{bib:gpl}
    \textit{Licencia GPL:}
    \url{https://www.gnu.org/licenses/gpl-3.0.en.html}
    
    \bibitem{bib:gltf}
    \textit{Documentación sobre glTF:}
    \url{https://github.com/KhronosGroup/glTF}
    
    \bibitem{bib:sketchfab}
    \textit{Librería con modelos 3D:}
    \url{https://sketchfab.com/}
    
    \bibitem{bib:alvaro}
    Álvaro Paniagua Tena.
    \textit{Repositorio de Websim 1.0}. URJC. Diciembre, 2018
    \url{https://github.com/RoboticsLabURJC/2018-tfg-alvaro_paniagua}
    
    \bibitem{bib:appsweb}
    \textit{Aplicaciones web:} 
    \url{https://wiboomedia.com/que-son-las-aplicaciones-web-ventajas-y-tipos-de-desarrollo-web/}
    
    \bibitem{bib:html5}
    Equipo desarrollador de Mozilla.
    \textit{Tecnología web para desarrolladores: HTML5}   \url{https://developer.mozilla.org/es/docs/HTML/HTML5}
     
    \bibitem{bib:middleware}
    \textit{Middlewares robóticos:}   \url{https://web.archive.org/web/2012062921151/http://www.middleware.org/whatis.html}
    
     \bibitem{bib:steam}
    \textit{Educación STEAM:}   \url{https://www.aulaplaneta.com/2018/01/15/recursos-tic/educacion-steam-la-integracion-clave-del-exito/}
    
    \bibitem{bib:programacionvisual}
    \textit{Entornos de Programación Visual para Programación Orientada a Objetos: Aceptación y Efectos en la Motivación de los Estudiantes. Felipe I. Anfarrutia, Ainhoa Álvarez, Mikel Larrañaga, Juan-Miguel López-Gil. 2017.}
    
    \bibitem{bib:extremeprogramming}
    \textit{Extreme programming: }
    \url{https://profile.es/blog/programador-extremo-extreme-programming/}
  
    \bibitem{bib:docjs}
    Equipo desarrollador de Mozilla.
    \textit{Tecnología web para desarrolladores: JavaScript}.
    \url{https://developer.mozilla.org/es/docs/Web/JavaScript}

    \bibitem{bib:aframe}
    \textit{Documentación A-Frame:}
    \url{https://aframe.io/}
    
    \bibitem{bib:fisicas}
    \textit{Físicas en A-Frame: }
    \url{https://github.com/donmccurdy/aframe-physics-system}
    
    \bibitem{bib:animaciones}
    \textit{Animaciones en A-Frame: }
    \url{https://blog.prototypr.io/learning-a-frame-how-to-do-animations-2aac1ae461da}
    
    \bibitem{bib:extras}
    \textit{A-Frame extras: }
    \url{https://github.com/donmccurdy/aframe-extras/tree/master/src/loaders}
    
    \bibitem{bib:blockly}
    \textit{Documentación Blockly: }
    \url{https://developers.google.com/blockly}
    
    \bibitem{bib:wwwschools}
    \textit{Desarrollo web: }
    \url{https://www.w3schools.com/}
    
    \bibitem{bib:blender}
    \textit{Documentación Blender: }
    \url{https://docs.blender.org/manual}
    
    \bibitem{bib:blenderinfo}
    \textit{Información sobre Blender: }
    \url{https://www.desarrollolibre.net/blog/blender/que-es-blender}
    
    \bibitem{bib:blendermodelado}
    \textit{Manual de modelado y animación con Blender. Pablo Suau. Universidad de Alicante. 2011.}
    
    \bibitem{bib:gltf}
    \textit{Información sobre glTF: }
    \url{https://www.khronos.org/gltf/}
   
   \bibitem{bib:kibotics}
    \textit{Kibotics:}  
    \url{https://www.kibotics.org/}

\end{thebibliography}