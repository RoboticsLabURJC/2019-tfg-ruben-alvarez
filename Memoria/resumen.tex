\chapter*{Resumen}
\markboth{RESUMEN}{RESUMEN} 

     Este proyecto está muy relacionado con las tecnologías web y la robótica y está enfocado a la mejora de la plataforma \emph{Kibotics} en la cual se dispone de un simulador robótico que tiene todo su peso computacional en el navegador web. 
       
    Esta plataforma está enfocada a enseñar robótica y programación a estudiantes desde muy corta edad hasta estudiantes de secundaria, empleando el lenguaje \emph{Scratch} en cortas edades y \emph{Python} en más avanzadas. Con ella, además de ejecutar el código programado, se puede enviar el mismo a un robot físico haciendo la traducción a \emph{Python}. \\
    
    
    Como aportes genuinos de este TFG se han añadido a \textit{Kibotics} soporte a nuevos robots tales como drones o mBot, nuevos ejercicios a partir de la funcionalidad existente y ejercicios competitivos para poder programar dos inteligencias en el mismo escenario. Estos ejercicios incorporan evaluadores automáticos que evaluan el comportamiento de los \textit{robots} simulados y con ello la eficacia del código del estudiante. Además se ha desarrollado una página web para poder teleoperar los robots sin necesidad de programarlos y comprobar sus sensores y actuadores. \\
    
    
    Para el desarrollo de software de todas las mejoras se han empleado herramientas como \textit{A-Frame}, \textit{HTML5}, \textit{JavaScript}, \textit{Blender} o \textit{Blockly}. Para la gestión de dependencias del proyecto se utiliza NPM y para el empaquetado de la aplicación, WebPack.