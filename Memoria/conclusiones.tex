\chapter{Conclusiones}
\label{chap:conclusiones}

Tras detallar las mejoras aportadas a \textit{WebSim}, en este capítulo se recopilan los objetivos alcanzados, se valoran los conocimientos adquiridos y se exponen las posibles líneas de mejora y extensión de la plataforma. 
\section{Conclusiones}

Repasando los objetivos establecidos en el capítulo \ref{chap:objetivos} se puede concluir que se han conseguido llevar a cabo todos los puntos establecidos. 

El primer objetivo consistía en añadir soporte a \textit{drone}  en \textit{WebSim}. En la sección \ref{sec:drone} se explica como se ha llevado a cabo este objetivo aportando el modelo en 3D, drivers y bloques para las nuevas funciones. \\

El segundo objetivo era incluir más ejercicios a \textit{WebSim}. Los nuevos escenarios se han explicado en la sección \ref{sec:escenarios} y otorga a la plataforma el poder realizar nuevos ejercicios como choca-gira (subsección \ref{subsec:chocagira}), sigue-pelota (subsecciones \ref{subsec:pelotapibot} y \ref{subsec:pelotadrone}) y atraviesa-bosque (subsección \ref{subsec:atraviesabosque}).  \\

El tercer objetivo consistía en añadir teleoperadores, que se explica en la sección \ref{sec:teleoperadores} y, además, se han creado archivos de configuración para poder cambiar de escenario y facilitar así su integración en servidor. En esta sección también se muestran los nuevos modelos creados (Fórmula 1 y mBot). \\

El último objetivo, ejercicios competitivos, se ha llevado a cabo en la sección \ref{sec:competitive}. Se han incorporado dos robots a este tipo de ejercicios (a excepción del ejercicio gato-ratón) incluyendo un nuevo modelo de Fórmula 1 y de \textit{drone} y además, se ha incorporado un evaluador automático por cada ejercicio. Estos evaluadores se han realizado  \\

Además, en el desarrollo de este proyecto, se ha llevado a cabo una refactorización de \textit{WebSim} actualizando así la aplicación a la versión \textit{WebSim 2.0}. En esta versión se separan los hilos de \textit{HAL API}, simulador y editor dando la posibilidad de crear más de un robot en la misma escena, cambiar el código del robot en la simulación o, incluso, parar la simulación y reanudarla después con un código distinto. Se ha formado parte de ella con elementos como \textit{brains}, que ejecuta la inteligencia de los \textit{robots} que haya en el mundo; refactorizando los editores disponibles (editor \textit{JavaScript}, editor \textit{Scratch}, editor competitivo \textit{JavaScript}, editor competitivo \textit{Scratch} y teleoperadores) para su correcto funcionamiento o realizando pruebas y ajustes para optimizar el rendimiento de la aplicación.

\section{Mejoras futuras}


Como mejoras futuras se pueden albergar las siguientes:
\begin{itemize}
    \item Añadir nuevos modelos de \textit{robots} como la aspiradora \textit{Roomba} o un \textit{robot} con pinzas. 
    \item Añadir más escenarios y ejercicios a la plataforma, por ejemplo aparcamiento automático o uno basado en coger objetos del entorno. 
    \item Explorar el uso de \textit{WebWorkers} para optimizar el rendimiento de \textit{WebSim}.
    \item Establecer un control en posición modificando la arquitectura de cómputo.
\end{itemize}